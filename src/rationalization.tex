\begin{center}
\bfseries{\large МАТЕРИАЛЫ ПО РАЦИОНАЛИЗАТОРСКИМ ПРЕДЛОЖЕНИЯМ}
\end{center}

В коде кролера стоит проверить необходимость конструкций try-except-finally: есть потенциально опасные места, где кролер может сломаться, и наоборот, где эта проверка замедляет работу кролера. Возможно, стоит поискать другие решения по отсеиванию уже пройденных страниц(есть вероятность, что в архиве страницы не повторяются и тогда роль pages = set() становиться сомнительной).

База данных состоит из новостей за последние 30 дней. Для более точных показателей модели, стоит запустить кролера на большее число дней. И стоит доработать новости с видеозаписями(проблема заключается в том, что tag с текстом новости другой в подобных страницах и это приводит к вылету программы с ошибкой).

Хотя вероятность попадания на страницы, которые запрещены в https://lenta.ru/robots.txt мала, лучше напрямую запретить эти страницы парсить(User-agent: GoogleBot, Disallow: /search, Disallow: /check\_ed и тд).

Так как на написание модели оставалось мало времени не удалось должным образом продумать ее параметры. Из-за того же недостатка было получено мало данных для обучения. Стоит проверить результаты модели на разных метриках и использование других оптимизаторов. Увеличение или уменьшение слоев и изменение их параметров(количество нейронов, функции активациии тд). Проверить результаты на разных алгоритмах.

\pagebreak
