\documentclass[dvipsnames,pdf, unicode, 12pt, a4paper, oneside, fleqn]{article}
\usepackage[utf8]{inputenc}
\usepackage[T2B]{fontenc}
\usepackage[english,russian]{babel}

\usepackage{listings}
\usepackage{longtable}
\oddsidemargin=-0.4mm
\textwidth=160mm
\topmargin=4.6mm
\textheight=210mm
\usepackage{geometry}
%% Страницы диссертациия должны иметь следующие поля:
%% левое --- 25 мм, правое --- 10 мм, верхнее --- 20 мм, нижнее --- 20 мм.
%% Абзацный отступ должен быть одинаковым по всему тексту и равен пяти знакам.
\geometry{
 a4paper,
 total={170mm,257mm},
 right=10mm,
 left=10mm,
 top=20mm,
 bottom=20mm,
}
\pagenumbering{gobble}

\usepackage{multicol}
\usepackage[]{amsmath}
\usepackage{multirow}

% THIS IS MY NEWLY DEFINED COMMAND
\newcommand\tline[2]{$\underset{\text{#1}}{\text{\underline{\hspace{#2}}}}$}

\usepackage{csquotes}
\DeclareQuoteStyle{russian}
    {\guillemotleft}{\guillemotright}[0.025em]
    {\quotedblbase}{\textquotedblleft}
\ExecuteQuoteOptions{style=russian}

\usepackage{longtable,array}

\newcolumntype{C}[1]{>{\centering\arraybackslash}p{#1}}
\setlength{\extrarowheight}{10pt}

\begin{document}

\begin{titlepage}
\begin{center}
\bfseries{\Large Министерство образования и науки\\Российской Федерации}

\vspace{12pt}

\bfseries{\Large Московский авиационный институт\\ (национальный исследовательский университет)}

\vspace{48pt}


%{\large Факультет информационных технологий и прикладной математики}

\vspace{36pt}


%{\large Кафедра вычислительной математики и~программирования}

\vspace{48pt}

{\huge ЖУРНАЛ}

\vspace{12pt}

{\large ПО ПРОИЗВОДСТВЕННОЙ ПРАКТИКЕ}


\end{center}

\vspace{72pt}

\begin{flushleft}
Наименование практики: {\itshape исследовательская}\\
Студент: И.\,Т. Батяновский \\
Факультет №8, курс 3, группа 7 \\
\end{flushleft}

\vspace{12pt}

\begin{flushleft}
Практика с 29.06.20 по 12.07.20
\end{flushleft}

\vfill

\begin{center}
\bfseries Москва, \the\year
\end{center}
\end{titlepage}

\pagebreak

\begin{center}
\bfseries{\large ИНСТРУКЦИЯ }

\vspace{12pt}

\bfseries{о заполнении журнала по производственной практике}
\end{center}

\begin{multicols}{2}
{\small
Журнал по производственной практике студентов имеет единую форму для всех видов практик.

Задание в журнал вписывается руководителем практики от института в первые три-пять дней пребывания студентов на практике в соответствии с тематикой, утверждённой на кафедре до начала практики. Журнал по производственной практике является основным документом для текущего и итогового контроля выполнения заданий, требований инструкции и программы практики.

Табель прохождения практики, задание, а также технический отчёт выполняются каждым студентом самостоятельно.

Журнал заполняется студентом непрерывно в процессе прохождения всей практики и регулярно представляется для просмотра руководителям практики. Все их замечания подлежат немедленному выполнению.

В разделе «Табель прохождения практики» ежедневно должно быть указано, на каких рабочих местах и в качестве кого работал студент. Эти записи проверяются и заверяются цеховыми руководителями практики, в том числе мастерами и бригадирами. График прохождения практики заполняется в соответствии с графиком распределения студентов по рабочим местам практики, утверждённым руководителем предприятия.
В разделе «Рационализаторские предложения» должно быть приведено содержание поданных в цехе рационализаторских предложений со всеми необходимыми расчётами и эскизами. Рационализаторские предложения подаются индивидуально и коллективно.

Выполнение студентом задания по общественно-политической практике заносятся в раздел «Общественно-политическая практика». Выполнение работы по оказанию практической помощи предприятию (участие в выполнении спецзаданий, работа сверхурочно и т.п.) заносятся в раздел журнала «Работа в помощь предприятию» с последующим письменным подтверждением записанной работы соответствующими цеховыми руководителями.
Раздел «Технический отчёт по практике» должен быть заполнен особо тщательно. Записи необходимо делать чернилами в сжатой, но вместе с тем чёткой и ясной форме и технически грамотно. Студент обязан ежедневно подробно излагать содержание работы, выполняемой за каждый день. Содержание этого раздела должно отвечать тем конкретным требованиям, которые предъявляются к техническому отчёту заданием и программой практики. Технический отчёт должен показать умение студента критически оценивать работу данного производственного участка и отразить, в какой степени студент способен применить теоретические знания для решения конкретных производственных задач.

Иллюстративный и другие материалы, использованные студентом в других разделах журнала, в техническом отчёте не должны повторяться, следует ограничиваться лишь ссылкой на него. Участие студентов в производственно-технической конференции, выступление с докладами, рационализаторские предложения и т.п. должны заноситься на свободные страницы журнала.

{\bfseries Примечание.} Синьки, кальки и другие дополнения к журналу могут быть сделаны только с разрешения администрации предприятия и должны подшиваться в конце журнала.

Руководители практики от института обязаны следить за тем, чтобы каждый цеховой руководитель практики перед уходом студентов из данного цеха в другой цех вписывал в журнал студента отзывы об их работе в цехе.

Текущий контроль работы студентов осуществляется руководители практики от института и цеховыми руководителями практики заводов. Все замечания студентам руководители делают в письменном виде на страницах журнала, ставя при этом свою подпись и дату проверки.

Результаты защиты технического отчёта заносятся в протокол и одновременно заносятся в ведомость и зачётную книжку студента.

{\bfseries Примечание.} Нумерация чистых страниц журнала проставляется каждым студентом в своём журнале до начала практики.
}
\end{multicols}

\begin{center}
С инструкцией о заполнении журнала ознакомились:
\end{center}

«\hspace{0.5cm}» \tline{(дата)}{1.5in} \the\year\,г.\hfillСтудент Батяновский И.\,Т. \tline{(подпись)}{1in}
\pagebreak

\begin{center}
\bfseries{\large ЗАДАНИЕ}
\end{center}

кафедры 806 по вычислительной/исследовательской практике:

парсить новостные ленты, новости, названия и модель, которая их классифицирует.

\vspace*{\fill}
Руководитель практики от института: 

\vspace{5pt}
\enquote{\hspace{0.5cm}} \tline{(дата)}{1.5in} \the\year\,г.\hfillКухтичев А.\,A. \tline{(подпись)}{1in}
\pagebreak
\begin{center}
\bfseries{\large ТАБЕЛЬ ПРОХОЖДЕНИЯ ПРАКТИКИ}
\end{center}

\begin{longtable}{|C{2cm}|C{6cm}|C{1.7cm}|C{1.5cm}|C{1.5cm}|C{2.8cm}|}
    \hline
    {\bfseries Дата} & {\bfseries Содержание или наименование проделанной работы} & {\bfseries Место работы} & \multicolumn{2}{c|}{{\bfseries Время работы}} & {\bfseries Подпись цехового руководителя}\\
    \cline {4-5} & & & Начало & Конец & \\
    \endfirsthead
    \hline
    {\bfseries Дата} & {\bfseries Содержание или наименование проделанной работы} & {\bfseries Место работы} & \multicolumn{2}{c|}{{\bfseries Время работы}} & {\bfseries Подпись цехового руководителя}\\
    \cline {4-5} & & & Начало & Конец & \\
    \hline
    \endhead
    \multicolumn{6}{c}{\textit{Продолжение на следующей странице}}
    \endfoot
    \endlastfoot
    \hline
    29.06.2019 & Получение задания & МАИ & 9:00 & 18:00 & \\
    \hline
    01.07.2019 & Чтение литературы по Web-scraping & МАИ & 9:00 & 18:00 & \\
    \hline
    02.07.2019 & Установка необходимого ПО & МАИ & 9:00 & 18:00 & \\
    \hline
    03.07.2019 & Изучение html страниц https://meduza.io/ & МАИ & 9:00 & 18:00 & \\
    \hline
    04.07.2019 & Написание парсера & МАИ & 9:00 & 18:00 & \\
    \hline
    05.07.2019 & Написание кролера(crawler) & МАИ & 9:00 & 18:00 & \\
    \hline
    06.07.2019 & Изучение html страниц https://lenta.ru/ & МАИ & 9:00 & 18:00 & \\
    \hline
    07.07.2019 & Написание парсера и кролера(с загрузкой данных на MySQL) & МАИ & 9:00 & 18:00 & \\
    \hline
    09.07.2019 & Поиск информации по написанию модели классификации текста & МАИ & 9:00 & 18:00 & \\
    \hline
    10.07.2019 & Написание программы перевода данных с MySQL в дата фрейм pandas & МАИ & 9:00 & 18:00 & \\
    \hline
    11.07.2019 & Написание модели & МАИ & 9:00 & 18:00 & \\
    \hline
    12.07.2018 & Сдача журнала & МАИ & 9:00 & 18:00 &  \\
    \hline
\end{longtable}

\pagebreak

\begin{center}
\bfseries{\large Отзывы цеховых руководителей практики}
\end{center}
Студент Батяновский И.\,Т. разработал веб кролер(внутри него парсер и загрузка данных на MySQL) и модель классификации текста.

Презентация защищена на комиссии кафедры 806. Работа выполнена в полном объёме. Рекомендую на оценку \enquote{\hspace{2cm}}. Все материалы сданы на кафедру.
\pagebreak


\begin{center}
\bfseries{\large ПРОТОКОЛ }

\vspace{12pt}

\bfseries{ЗАЩИТЫ ТЕХНИЧЕСКОГО ОТЧЁТА}
\end{center}
\noindent
по {\itshapeпроизводственной практике}

\vspace{8pt}
\noindent
студентами:
\noindent
Батяновский Иван Тарасович

\begin{longtable}{p{7cm}|p{11cm}}
    \hline
    {\bfseries Слушали:} & {\bfseries Постановили:}  \\
    \endfirsthead
    \hline
    {\bfseries Слушали:} & {\bfseries Постановили:}  \\
    \hline
    \endhead
    \multicolumn{2}{c}{\textit{Продолжение на следующей странице}}
    \endfoot
    \endlastfoot
    Отчёт практиканта & считать практику выполненной и защищённой на\\
    \rule{0pt}{425pt} & Общая оценка: \underline{\hspace{2in}}\\
    \rule{0pt}{15pt} & \\
    \hline
\end{longtable}

\vfill

\noindent\begin{tabular}{@{}l l l}
Руководители: & Зайцев В.\,Е. & \underline{\hspace{2in}}\\
 \rule{0pt}{10pt} & Кухтичев А.\,А. & \underline{\hspace{2in}}
\end{tabular}
\vspace{12pt}

\noindent
Дата: 12 июля \the\year\,г.
\pagebreak

\begin{center}
\bfseries{\large МАТЕРИАЛЫ ПО РАЦИОНАЛИЗАТОРСКИМ ПРЕДЛОЖЕНИЯМ}
\end{center}

В коде кролера стоит проверить необходимость конструкций try-except-finally: есть потенциально опасные места, где кролер может сломаться, и наоборот, где эта проверка замедляет работу кролера. Возможно, стоит поискать другие решения по отсеиванию уже пройденных страниц(есть вероятность, что в архиве страницы не повторяются и тогда роль pages = set() становиться сомнительной).

База данных состоит из новостей за последние 30 дней. Для более точных показателей модели, стоит запустить кролера на большее число дней. И стоит доработать новости с видеозаписями(проблема заключается в том, что tag с текстом новости другой в подобных страницах и это приводит к вылету программы с ошибкой).

Хотя вероятность попадания на страницы, которые запрещены в https://lenta.ru/robots.txt мала, лучше напрямую запретить эти страницы парсить(User-agent: GoogleBot, Disallow: /search, Disallow: /check\_ed и тд).

Так как на написание модели оставалось мало времени не удалось должным образом продумать ее параметры. Из-за того же недостатка было получено мало данных для обучения. Стоит проверить результаты модели на разных метриках и использование других оптимизаторов. Увеличение или уменьшение слоев и изменение их параметров(количество нейронов, функции активациии тд). Проверить результаты на разных алгоритмах.

\pagebreak




\begin{center}
\bfseries{\large ТЕХНИЧЕСКИЙ ОТЧЁТ ПО ПРАКТИКЕ}
\end{center}

\section*{Архитектура}
\begin{lstlisting}[language=Python]
from keras.models import Sequential
from keras import layers

embedding_dim = 50

model2 = Sequential()
model2.add(layers.Embedding(input_dim=vocab_size, 
                           output_dim=embedding_dim, 
                           input_length=maxlen))
model2.add(layers.GlobalMaxPool1D())
model2.add(layers.Dense(256, activation='relu'))
model2.add(layers.Dense(11, activation='softmax'))
model2.compile(optimizer='adam',
              loss='categorical_crossentropy',
              metrics=['accuracy'])
model2.summary()
\end{lstlisting}
\section*{Описание}

Последовательность запускаемых файлов и их смысл:
\begin{enumerate}
    \item lentaScrapper.py
    
    Эта программа запускает кролера, который просматривает определенное количество страниц в архиве lenta.ru с разных рубрик. В целом никаких параметров изменять не надо, кроме одного - limit. limit - количество дней, которые нужно просмотреть в ленте начиная с текущего.
    
    Что насчет описания работы самого кролера, то страница архива ленты представляет собой(имеется ввиду где нужная информация находится) 3 блока span4, в которой назодятся гиперссылки на статья. 
    
    Для получения ссылок на архив разных рубрик программа парсит эти ссылки с главной страницы. Конкатенируя ссылки на рубрики и текущую дату, получаем ссылку на страницу архива, через которую сканируем ссылки на статьи и проходим по архиву через кнопку <, которая хранит ссылку на предыдущий день. Поэтому в теории можно пройтись по всем статьям сайта, но моей главной задачей стояла прежде всего получить опыт в написании кролера, а не полный сбор данных с сайта(кроме того это замет слишком много времени).
    
    Для работы Selenium нужно для своего браузера гугл установить соот версию chromedriver.exe
    
    \item Save-.py
    
    Эта программа переводит данные с MySQL сервера в файл формата .csv, с которым придется в дальнейшем работать при обучении модели.
    
    \item output\_with\_sport.csv
    
    Итоговая таблица данных(id, title, genre, content)
    \begin{itemize}
        \item genre - это название рубрики. Это название стоило бы поменять на rubrics(genre - это унаследованное название с кролера, который я написал сначала для Meduza, однако я решил переделать его для Lenta, т.к. многие статьи в Meduza выложены без конкретного описания тематики)
    \end{itemize}
    
    \item classText.ipynb
    
    Ноутбук где последовательно обрабатывается текст, пишется модель, обучается и записаны результаты.
    
\end{enumerate}


\section*{Реализация}

\begin{lstlisting}[language=Python]
def crawler(url="", rubric="NULL", date=date.today(),
depth=0, limit=3, main=False):
    global pages
    if main is True:
        newUrl = "https://lenta.ru" + url + str(date)[:4] + "/" + str(date)[5:7] + "/" + str(date)[8:] + "/"
    else:
        newUrl = "https://lenta.ru" + url
    driver.get(newUrl)
    print(newUrl, rubric)
    try:
        # Waiting for block of news to appear
        element = WebDriverWait(driver, 10) \
            .until(EC.presence_of_element_located
            ((By.CLASS_NAME, "b-layout.js-layout.b-layout_archive")))
    except:
        print("Can't locate news block")
        return
    finally:
        bsObj = BeautifulSoup(driver.page_source, "html.parser")

    span4s = bsObj.find("section",
    {"class": "b-layout js-layout b-layout_archive"})
    .findAll("div", {"class": "span4"})
    for span4 in span4s:
        items = span4.findAll("div", 
        {"class": "item news b-tabloid__topic_news"})
        for item in items:
            if items is not None:
                newLink = item.find("a", {"href": re.compile("^(\/.*\/).*")})
                .attrs["href"]
                if newLink not in pages:
                    pages.add(newLink)
                    parser(newLink, rubric)
    if depth < limit:
        print("Push some buttons sometimes(dep, lim):", depth + 1, limit)
        crawler(url=bsObj.find("a", {"class": "control_mini"})
        .attrs["href"], rubric=rubric, depth=depth + 1, limit=limit, date=date)

\end{lstlisting}

\section*{Тестирование}

text = "В наступившем году должны состояться первые с июля 2011\го пилотируемые полеты США к МКС на собственных космических кораблях (до этого США отправляли своих астронавтов на околоземную орбиту при помощи многоразовых космических кораблей Space Shuttle). Скорее всего, первым из них в первом полугодии стартует Crew Dragon компании SpaceX, в декабре 2019-го успешно завершивший испытания парашютной системы, к которой ранее у НАСА были претензии, а до этого, в марте того же года, выполнивший первый (в беспилотном режиме) полет к МКС."

\begin{lstlisting}[language=Python]
re = tokenizer.texts_to_sequences([text])
resu = pad_sequences(re, padding='post', maxlen=maxlen)
otvet = model2.predict(resu)

print(np.where(otvet == np.amax(otvet))[1])
print(maper)

[6]
{"'russia'": 1, "'world'": 2, "'ussr'": 3, "'economics'": 4,
"'forces'": 5, "'science'": 6, "'culture'": 7, "'sport'": 8,
"'media'": 9, "'style'": 10}
\end{lstlisting}

В данном примере я взял новость из категории наука  https://lenta.ru/articles/2020/01/08/2020/. Ответ [6] означает, что данная новость относится к категории наука, что верно. 

\section*{Ссылка на GitHub}

https://github.com/Ivan-Batyanovsky/Practice2020

\pagebreak


\end{document}
